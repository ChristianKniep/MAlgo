\documentclass[11pt]{article}
\newcommand{\pns}{nichts\"attigende Pushes\ }
\newcommand{\ps}{s\"attigende Pushes\ }

\author{Christian Kniep}
\title{Methoden der Algorithmik WS1011 \\ Prof. Dr. Michael Kaufmann}
\date{\today}
%----------------------------
\begin{document}
\maketitle
\section{Vorlesung 10.11.2010}
\subsection{Wiederholung}
\begin{itemize}
    \item Anzahl \ps: $\leq n*m$
    \item Anzahl \pns: $\leq O(n^2*m)$
    \item Anzahl Relabel: $\leq 2n$
\end{itemize}

\subsection{Lemma}
Wir haben h\"ochstens $\leq O(n^2*m)$ \pns.
\subsection{Beweis}
Benutzen Potentialfunktion $\phi = \displaystyle\sum_{v \in A} d(v)$.(A=Menge aller aktiven Knoten) \\
Da $|A| \leq n, d(v) \leq$ f\"ur alle $v \ in V$ und $\phi \leq 2n^2$ anfangs.\\
Am Ende ist $\phi = 0$.
\begin{enumerate}
    \item \textbf{Relabel} \\
        $\phi$ steigt um $\leq \epsilon$, wenn  d(v) um $\epsilon$ steigt.\\
        Insgesamt steigt d(v) um $\leq 2n$, also insgesamt steigt $\phi$ um $\leq 2n^2$.
    \item \textbf{\ps} \\
        Diese erzeugen \"uber (v,w) vtl. neuen \"Uberfluss an w, w wird somit aktiv, $\phi$ steigt um $d(w) \leq 2n$.
        Also steigt $\phi$ durch \ps um $\leq 2n^2*m$.
    \item \textbf{\pns} \\
        $\phi$ wird um d(v) erniedrigt, evtl. jedoch noch um d(w) erh\"oht, falls w vorher nicht aktiv war. \\
        Es gilt jedoch $d(v) \geq d(w) + 1 \Rightarrow \phi$ erniedrigt sich um $\geq 1$.
\end{enumerate}
\paragraph{Insgesammt:} \ \\
Erh\"ohung von $\phi$ von $\leq 2n^2$ um $2n^2 + 2n^2*m$ und k Erniedrigungen von $\phi$ um $\geq 1$,\\
f\"uhren zusammen zu 0-Potential. \\
$\Rightarrow k = O(n^2*m), k =$ \# \pns

\subsection{Satz}
Der generische Preflow-Push-Algorithmus l\"auft in $O(n^2*m)$.
\subsection{Varianten} Wahl der aktiven Knoten.
\begin{enumerate}
    \item \textbf{FIFO} \\
        Warteschlange aktiver Knoten $O(n^3)$
    \item \textbf{Highest Label} \\
        'h\"ochster' Knoten im Netz $O(n^2\sqrt{m})$
    \item \textbf{Excess Scaling} \\
        'h\"ochster' F\"ullstand zuerst $O(n*m + log c)$, c=gr\"osste Kapazit\"at
\end{enumerate}
\subsection{FIFO}
\subsubsection{Regel}
Wende Push/Relabel solange auf denselben Knoten an, bis entweder e(v)=0 oder Relabel-Operation angewendet wurde.\\
DIe Liste der Knoten wird als FIFO-Queue gehalten.\\
Wird v relabelt, wird es hinten wieder angef\"ugt.
\subsubsection{Phasen}
Arbeiten die Liste ab, bis um ersten Mal ein Knoten erscheint, der an dieser Phase schon teilgenommen hat.
\subsubsection{Behauptung}
Es gibt $\leq 4n^2 + 2n$ Phasen
\subsubsection{Beweis}
Betrachte jeweils die \"Anderung der Potentialfunktion.\\
$\phi = max\{d(v)|v ist aktiv\}$
\begin{enumerate}
    \item \textbf{Fall 1} \\
        W\"ahrend einer Phase gibt es mindestens eine Relabel-Operation.\\
        $\phi$ steigt h\"ochstens so viel wie der d-Wert max. (?).
        $\Rightarrow \phi$ steigt in solchen F\"allen um $\leq 2n^2$.
    \item \textbf{Fall 2} \\
        Alle aktiven Knoten mit max. Distanz werden inaktiv, der maximale Distanzwert sinkt um mindestens 1.
\end{enumerate}
Insgesamt gibt es also $\leq \underbrace{n+2n^2 + (n+2n^2)}_{2n+4n^2}$ Pushes.
\subsubsection{Satz}
Der FIFO-Preflow-Push-Algorithmus l\"auft in $O(n^3)$

\subsection{Highest-Label}
\subsubsection{Regel}
Schicke immer Fluss von einem aktiven Knoten mit h\"ochstem Distanzwert.
\begin{enumerate}
    \item Es gibt $O(n^3)$ \pns.\\
        Sei $h = max\{d(v)|v aktiv\}$\\
        Zuerst werden aktive Knoten v mit $d(v)=h$ betrachtet, \\
        dann $h-1, h-2, usw.$
        Bei Relabelings f\"angt das ganze neu an.\\
        Es gibt jedoch nur $\leq 2n^2$ Relabelings.\\
        Gibt es n \pns hintereinander, dann sind wir fertig. Alle Knoten sind dann inaktiv.
\end{enumerate}
Wie findet man aktive Knoten mit h\"ochstem DIstanzlabel? \\
Halten Liste $(k) = {v | v aktiv und d(v)=k}$.\\
Merken max. Index der den h\"ochsten d-Wert angibt. Betrachte Listen $(max)$, Listen $(max-1)$, usw...\\
Relabel erh\"oht max.
\subsection{Satz}
Der Highest-Label-Preflow-Push macht $O(n^2\sqrt{m}$ \pns und hat somit $O(n^3)$ Gesamtlaufzeit.
\end{document}