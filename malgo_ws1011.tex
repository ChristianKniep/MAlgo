\documentclass[11pt]{article}
\usepackage{txfonts}
\usepackage{verbatim}

\newcommand{\pns}{nichts\"attigende Pushes\ }
\newcommand{\ps}{s\"attigende Pushes\ }
\newcommand{\A}{Algorithmus}
\newcommand{\RA}{Randomisierter Algorithmus}
\newcommand{\z}{zuf\"allig\ }
\newcommand{\mhW}{mit hoher Wahrscheinlichkeit (m.h.W.)\ }
\newcommand{\MCA}{Monte-Carlo Algorithm}
\newcommand{\R}{\varmathbb{R}}

\author{Christian Kniep}
\title{Methoden der Algorithmik WS1011 \\ Prof. Dr. Michael Kaufmann}
\date{\today}
%----------------------------
\begin{document}
\maketitle
\section{Vorlesung 10.11.2010}
\subsection{Wiederholung}
\begin{itemize}
    \item Anzahl \ps: $\leq n*m$
    \item Anzahl \pns: $\leq O(n^2*m)$
    \item Anzahl Relabel: $\leq 2n$
\end{itemize}

\subsection{Lemma}
Wir haben h\"ochstens $\leq O(n^2*m)$ \pns.
\subsection{Beweis}
Benutzen Potentialfunktion $\phi = \displaystyle\sum_{v \in A} d(v)$.(A=Menge aller aktiven Knoten) \\
Da $|A| \leq n, d(v) \leq$ f\"ur alle $v \ in V$ und $\phi \leq 2n^2$ anfangs.\\
Am Ende ist $\phi = 0$.
\begin{enumerate}
    \item \textbf{Relabel} \\
        $\phi$ steigt um $\leq \epsilon$, wenn  d(v) um $\epsilon$ steigt.\\
        Insgesamt steigt d(v) um $\leq 2n$, also insgesamt steigt $\phi$ um $\leq 2n^2$.
    \item \textbf{\ps} \\
        Diese erzeugen \"uber (v,w) vtl. neuen \"Uberfluss an w, w wird somit aktiv, $\phi$ steigt um $d(w) \leq 2n$.
        Also steigt $\phi$ durch \ps um $\leq 2n^2*m$.
    \item \textbf{\pns} \\
        $\phi$ wird um d(v) erniedrigt, evtl. jedoch noch um d(w) erh\"oht, falls w vorher nicht aktiv war. \\
        Es gilt jedoch $d(v) \geq d(w) + 1 \Rightarrow \phi$ erniedrigt sich um $\geq 1$.
\end{enumerate}
\paragraph{Insgesammt:} \ \\
Erh\"ohung von $\phi$ von $\leq 2n^2$ um $2n^2 + 2n^2*m$ und k Erniedrigungen von $\phi$ um $\geq 1$,\\
f\"uhren zusammen zu 0-Potential. \\
$\Rightarrow k = O(n^2*m), k =$ \# \pns

\subsection{Satz}
Der generische Preflow-Push-Algorithmus l\"auft in $O(n^2*m)$.
\subsection{Varianten} Wahl der aktiven Knoten.
\begin{enumerate}
    \item \textbf{FIFO} \\
        Warteschlange aktiver Knoten $O(n^3)$
    \item \textbf{Highest Label} \\
        'h\"ochster' Knoten im Netz $O(n^2\sqrt{m})$
    \item \textbf{Excess Scaling} \\
        'h\"ochster' F\"ullstand zuerst $O(n*m + log c)$, c=gr\"osste Kapazit\"at
\end{enumerate}
\subsection{FIFO}
\subsubsection{Regel}
Wende Push/Relabel solange auf denselben Knoten an, bis entweder e(v)=0 oder Relabel-Operation angewendet wurde.\\
DIe Liste der Knoten wird als FIFO-Queue gehalten.\\
Wird v relabelt, wird es hinten wieder angef\"ugt.
\subsubsection{Phasen}
Arbeiten die Liste ab, bis um ersten Mal ein Knoten erscheint, der an dieser Phase schon teilgenommen hat.
\subsubsection{Behauptung}
Es gibt $\leq 4n^2 + 2n$ Phasen
\subsubsection{Beweis}
Betrachte jeweils die \"Anderung der Potentialfunktion.\\
$\phi = max\{d(v)|v ist aktiv\}$
\begin{enumerate}
    \item \textbf{Fall 1} \\
        W\"ahrend einer Phase gibt es mindestens eine Relabel-Operation.\\
        $\phi$ steigt h\"ochstens so viel wie der d-Wert max. (?).
        $\Rightarrow \phi$ steigt in solchen F\"allen um $\leq 2n^2$.
    \item \textbf{Fall 2} \\
        Alle aktiven Knoten mit max. Distanz werden inaktiv, der maximale Distanzwert sinkt um mindestens 1.
\end{enumerate}
Insgesamt gibt es also $\leq \underbrace{n+2n^2 + (n+2n^2)}_{2n+4n^2}$ Pushes.
\subsubsection{Satz}
Der FIFO-Preflow-Push-Algorithmus l\"auft in $O(n^3)$

\subsection{Highest-Label}
\subsubsection{Regel}
Schicke immer Fluss von einem aktiven Knoten mit h\"ochstem Distanzwert.
\begin{enumerate}
    \item Es gibt $O(n^3)$ \pns.\\
        Sei $h = max\{d(v)|v aktiv\}$\\
        Zuerst werden aktive Knoten v mit $d(v)=h$ betrachtet, \\
        dann $h-1, h-2, usw.$
        Bei Relabelings f\"angt das ganze neu an.\\
        Es gibt jedoch nur $\leq 2n^2$ Relabelings.\\
        Gibt es n \pns hintereinander, dann sind wir fertig. Alle Knoten sind dann inaktiv.
\end{enumerate}
Wie findet man aktive Knoten mit h\"ochstem DIstanzlabel? \\
Halten Liste $(k) = {v | v aktiv und d(v)=k}$.\\
Merken max. Index der den h\"ochsten d-Wert angibt. Betrachte Listen $(max)$, Listen $(max-1)$, usw...\\
Relabel erh\"oht max.
\subsection{Satz}
Der Highest-Label-Preflow-Push macht $O(n^2\sqrt{m}$ \pns und hat somit $O(n^3)$ Gesamtlaufzeit.

\section{Vorlesung 17.11.2010}
\subsection{\"Uberblick randomisierte Algorithmen}
\subsubsection{Definition} Ein \A der Entscheidungen zuf\"allig trifft heisst \RA
\subsubsection{Vorteile}
\begin{itemize}
    \item Die \RA sind oft schneller als die deterministischen.
    \item sind (oft) viel einfacher
    \item keine oder wenige Worst-Case Eingaben
\end{itemize}
Es gibt zwei Typen von \RA:
\begin{enumerate}
    \item {\textbf Las-Vegas-Algorithmen} Liefern immer das korrekte Ergebnis! \\
        Durch \z getroffene Entscheidung variiert die Laufzeit.\\
        Ziel der Analyse: Laufzeit des LV Algorithmus ist \mhW $\leq X$.
    \item {\textbf Monte-Carlo Algorithmus} K�nnen falsches Ergebnis liefern! \\
        Ziel der Analyse: Wahrscheinlichkeit f\"ur falsches Ergebnis $\leq X$ \mhW .
\end{enumerate}
F\"ur Entscheidungsprobleme gibt es 2 Arten von \MCA en.
\begin{itemize}
    \item {\textbf einseitiger Fehler}: Algorithmus kann sich nur in eine Richtung irren. \\
        Z.B. ist Ergebnis von Algo. 'Ja' ist dies immer korrekt (Fehlerwahrscheinlichkeit ist 0).\\
        ist Ergebnis 'Nein', dann ist die Fehlerwahrscheinlichkeit $\geq 0$.
    \item {\textbf beidseitiger Fehler} Algorithmus kann sich in beide Richtungen irren. \\
        Fehlerwahrscheinlichkeit in beiden F\"allen $\geq o$.
\end{itemize}
\subsubsection{Wahrscheinlichkeitstheorie}
Sei $\Omega$ Ereignisraum. Ein Ereignis ist ELement aus $\Omega$. \\
Wahrscheinlichkeit ist Abbilsung von Ereignissen auf reelle Zahlen, so dass die Summe \"uber alle Ereignisse $= 1$. \\
$prob: \Omega \rightarrow \R^+_0 \sum_{w \in \Omega} prob(w) = 1$ \\
Ist $\Omega$ endlich und es gilt $prob(w) = \frac{1}{|\Omega|}$ so heisst prob Gleichverteilung.

\subsubsection{Beispiel}
$\Omega = {1,2,3,4,5,6}$ und $prob(w) = \frac{1}{6}$ f\"ur alle $w \in \Omega$.
$\Omega_{gerade} = {2,4,6}$ und $prob(w) \in \Omega_{gerade} = 3 \frac{1}{6} = \frac{1}{2}$
\subsubsection{Markovsche Ungleichung}
F\"ur bestimmtes Ereignis X und dessen Erwartungswert E(X) gilt:\\
$prob(X \geq k*E(X)) \leq \frac{1}{k}$

\newpage
\subsection{Random Walk}
BILD \\
\subsubsection{Frage}
$t_n(i) = E(Anzahl Schritte bis Abgrund erreicht) $ f\"ur $n=Feld,i=Position der Person$
\subsubsection{Analyse}
\paragraph{I} \ \\
$t(0) = 0$ \\
$t(1) = 1 + \frac{1}{2} t(0) + \frac{1}{2} t(2)$ \\
$t(2) = 1 + \frac{1}{2} t(1) + \frac{1}{2} t(3)$ \\
. \\
. \\
$t(n-1) = 1 + \frac{1}{2} t(n-2) + \frac{1}{2} t(n)$
$t(n) = 1 + t(n-1)$
\paragraph{II} \ \\
$t(n-1) \leq 1 + \frac{1}{2} t(n-2) + \frac{1}{2} [ 1 + t(n-1) ] = \frac{3}{2} + \frac{1}{2} t(n-2) + \frac{1}{2} t(n-1)$ \\
Gleichung $*2*t(n-1)$:\\
$t(n-1) \leq 3 + t(n-2)$ \\
$t(n-2) \leq 1 + \frac{1}{2} t(n-3) + \frac{1}{2} [3 + t(n-2)] = \frac{5}{2} + \frac{1}{2} t(n-3) + \frac{1}{2} t(n-2)$\\
$ \leq 5 + t(n-3)$\\
Allgemein: \\
$t(i) \leq 2(n-1) + 1 + t(i-1)$ \\
$t(2) \leq 2(n-2) + 1 + t(1) = 2n -3 + t(1)$ \\
$t(1) \leq 2(n-1) + 1 + \underbrace{t(0)}_{0} = 2n-1$
\paragraph{III}\ \\
$t(1) \leq 2n -1$\\
$t(2) \leq 2n -3 +  2n - 1 = 4n - 4$ \\
$t(3) \leq 2n -5 + 4n - 4 = 6n - 9$ Vermutung: $t(i) \leq 2in -i^2$ \\
\paragraph{Vollst\"qndige Induktion \"uber i} \ \\
IA: gezeigt f\"ur $i = 0,1,2,3$\\
IS: $t(i+1) \leq 2(n-(i+1)) + 1 + t(i)$\\
$\leq 2n -2i -1 +  2ni - i^2$\\
$\leq 2n(i+1) - (i+1)^2 \ _\square$ \\
\\
$\Rightarrow$ Also gilt: $t_n(n) \leq 2n^2 - n^2 = n^2$\\
Klar ist, dass $t_n(i-1) \leq t_n(i) \forall i$, daher ist $t_1(i) \leq n^2 \forall i \leq n$ \\
$\Rightarrow$  Random Walk terminiert in $O(n^2)$ Schritten \mhW
\\
\\
Einfache Anwendung von Random Walk: 2-SAT. \\
Gegeben: Boolscher Term in KNF, wobei alle Klauseln $\leq$ 2 Literale haben.\\
Beispiel: $ (a_1 \vee a_2) \wedge (\bar a_3 \vee a_4) \wedge (a_5) \wedge ...$
Gesucht: Variablenbelegung aller $a_i$ mit true/false, so dass alle Klauseln erf\"ullt sind.
A1: $a_1=t, a_2=f,a_3=t$\\
A2: $a_1=t, a_2=t,a_3=t$\\
Hamming Abstand von zwei Variablenbelegungen ist die Anzahl der Variablen an denen sich die Belegung unterscheidet.

\paragraph{Frage} Ist F erf\"ullbar? \\
2-SAT von Papadimitriou (1991)\\
\begin{enumerate}
    \item Rate Variablenbelegung $a_1 ... a_n =: A$
    \item Wiederhole (2$n^2$ Mal) {\\
        if (A erf\"ullt alle Klauseln) return true\\
        while \underline{zuf\"allig} eine Klausel C, die von A nicht erf\"ullt wird  \\
        und w\"ahle einen der beidenen Literale aus C und \underline{negiere} den Wert. \\
        $\Rightarrow$ dies ist neues A \\
    }
    \item return false; (\mhW \ nicht erf\"ullbar)
\end{enumerate}
\paragraph{Analyse}\ \\
(richtige Belegung $e$ ist 'Abgrund', $\bar e$ ist ganz links an der Wand und hat gr\"ossten Hamming Abstand.) \\
\\
Angenommen es flipt erf\"ullende Belegung $e$ (nur eine, falls mehr ist sogar besser => mehr Abgr\"unde) \\
$\bar e$ ist Belegung mit dem Hammingabstandn zu e (alle Variablen geflipt). \\
\begin{itemize}
    \item Da der Algo immer eine unerf\"ullte Klausel w\"ahlt ist klar, dass mindestens eines der beiden Literale den entgegengesetzten Wert haben muss.\\
        D.h. beim \"Andern des richtigen Literals kommt man e um einen Schritt n\"aher, verk\"urzt somit den Hammingabstand.
        $\Rightarrow prob$ hierf\"ur $\geq frac{1}{2}$.\\
    \item Belegung $\bar e$ (ganz links) f\"uhrt jeder Flip zu einer besseren Belegung.
\end{itemize}
$\Rightarrow$ entspricht Random Walk\\
X := bedeutet die Anzahl der Schleifendurchl\"aufe f\"ur erf\"ullbare Instanzen \\
$E(X) \leq n^2$ \\
Mit Mavkov-Ungleichung gilt: \\
$prob(X \geq k*E(X)) \leq \frac{1}{k}$ \\
In unserem Fall: \\
$prob(X \geq 2n^2) \leq \frac{1}{2}$ \\

$\Rightarrow$ Laufzeit ist O($n^2$) \\
Algo ist MonteCarlo Algo mit einseitigem Fehler. \\
Die Ausgabe 'false' stimmt nur mit Wahrscheinlichkeit $\geq \frac{1}{2}$

\end{document}